% Print marker from average data file instead of the last point
%
% Draw a solid line at the median and keep the variance semi-transparent
% Raw data should be marks only?
%
%
%


\documentclass[tikz]{standalone}
\usepackage{mathptmx}

\usepackage{varwidth}
\usepackage{pgfplots}
\pgfplotsset{compat=1.15}
\usepgfplotslibrary{units}
\usepgfplotslibrary{colorbrewer}

\usetikzlibrary{decorations.markings}
\usetikzlibrary{arrows,fit,calc,angles,quotes}

\usepackage{tikz-3dplot}
\usepackage{tkz-fct}
\usepackage{siunitx}
\usepackage{pgfplotstable}
\usepgfplotslibrary{groupplots}

\pgfplotsset{
    mark last*/.style={
        scatter,
        scatter src=x,
        scatter/@pre marker code/.code={
            \pgfmathtruncatemacro\usemark{
                (\coordindex==(\numcoords-1))
            }
            \ifnum\usemark=0
                \pgfplotsset{mark=none}
            \fi
        },
        scatter/@post marker code/.code={}
    }
}

\begin{filecontents}{avg.dat}
t
1
\end{filecontents}






\definecolor{Dark2-3-1}{RGB}{27,158,119}
\definecolor{Dark2-3-A}{RGB}{27,158,119}
\definecolor{Dark2-3-2}{RGB}{217,95,2}
\definecolor{Dark2-3-B}{RGB}{217,95,2}
\definecolor{Dark2-3-3}{RGB}{117,112,179}
\definecolor{Dark2-3-C}{RGB}{117,112,179}
\definecolor{Dark2-4-1}{RGB}{27,158,119}
\definecolor{Dark2-4-A}{RGB}{27,158,119}
\definecolor{Dark2-4-2}{RGB}{217,95,2}
\definecolor{Dark2-4-B}{RGB}{217,95,2}
\definecolor{Dark2-4-3}{RGB}{117,112,179}
\definecolor{Dark2-4-C}{RGB}{117,112,179}
\definecolor{Dark2-4-4}{RGB}{231,41,138}
\definecolor{Dark2-4-D}{RGB}{231,41,138}
\definecolor{Dark2-5-1}{RGB}{27,158,119}
\definecolor{Dark2-5-A}{RGB}{27,158,119}
\definecolor{Dark2-5-2}{RGB}{217,95,2}
\definecolor{Dark2-5-B}{RGB}{217,95,2}
\definecolor{Dark2-5-3}{RGB}{117,112,179}
\definecolor{Dark2-5-C}{RGB}{117,112,179}
\definecolor{Dark2-5-4}{RGB}{231,41,138}
\definecolor{Dark2-5-D}{RGB}{231,41,138}
\definecolor{Dark2-5-5}{RGB}{102,166,30}
\definecolor{Dark2-5-E}{RGB}{102,166,30}
\definecolor{Dark2-6-1}{RGB}{27,158,119}
\definecolor{Dark2-6-A}{RGB}{27,158,119}
\definecolor{Dark2-6-2}{RGB}{217,95,2}
\definecolor{Dark2-6-B}{RGB}{217,95,2}
\definecolor{Dark2-6-3}{RGB}{117,112,179}
\definecolor{Dark2-6-C}{RGB}{117,112,179}
\definecolor{Dark2-6-4}{RGB}{231,41,138}
\definecolor{Dark2-6-D}{RGB}{231,41,138}
\definecolor{Dark2-6-5}{RGB}{102,166,30}
\definecolor{Dark2-6-E}{RGB}{102,166,30}
\definecolor{Dark2-6-6}{RGB}{230,171,2}
\definecolor{Dark2-6-F}{RGB}{230,171,2}
\definecolor{Dark2-7-1}{RGB}{27,158,119}
\definecolor{Dark2-7-A}{RGB}{27,158,119}
\definecolor{Dark2-7-2}{RGB}{217,95,2}
\definecolor{Dark2-7-B}{RGB}{217,95,2}
\definecolor{Dark2-7-3}{RGB}{117,112,179}
\definecolor{Dark2-7-C}{RGB}{117,112,179}
\definecolor{Dark2-7-4}{RGB}{231,41,138}
\definecolor{Dark2-7-D}{RGB}{231,41,138}
\definecolor{Dark2-7-5}{RGB}{102,166,30}
\definecolor{Dark2-7-E}{RGB}{102,166,30}
\definecolor{Dark2-7-6}{RGB}{230,171,2}
\definecolor{Dark2-7-F}{RGB}{230,171,2}
\definecolor{Dark2-7-7}{RGB}{166,118,29}
\definecolor{Dark2-7-G}{RGB}{166,118,29}
\definecolor{Dark2-8-1}{RGB}{27,158,119}
\definecolor{Dark2-8-A}{RGB}{27,158,119}
\definecolor{Dark2-8-2}{RGB}{217,95,2}
\definecolor{Dark2-8-B}{RGB}{217,95,2}
\definecolor{Dark2-8-3}{RGB}{117,112,179}
\definecolor{Dark2-8-C}{RGB}{117,112,179}
\definecolor{Dark2-8-4}{RGB}{231,41,138}
\definecolor{Dark2-8-D}{RGB}{231,41,138}
\definecolor{Dark2-8-5}{RGB}{102,166,30}
\definecolor{Dark2-8-E}{RGB}{102,166,30}
\definecolor{Dark2-8-6}{RGB}{230,171,2}
\definecolor{Dark2-8-F}{RGB}{230,171,2}
\definecolor{Dark2-8-7}{RGB}{166,118,29}
\definecolor{Dark2-8-G}{RGB}{166,118,29}
\definecolor{Dark2-8-8}{RGB}{102,102,102}
\definecolor{Dark2-8-H}{RGB}{102,102,102}



\begin{document}

\def\figWidth{14cm}
\def\polMark{mark=pentagon*}
\def\baseMark{mark=x,}

\newcommand{\testLine}[2]{%
	\addplot+[%
		forget plot,
		color = Dark2-5-#2,
% 		only marks,
% 		opacity=0.5,
% 		mark options={scale=0.2},
		opacity=0.25,
	] table[%
		x=TimeInSeconds, 
		col sep=comma,
% 		y=MsUntilRenderComplete,
		y=MsBetweenPresents,
% 		y expr= (1000/(\thisrowno{11})),
	] {#1};
}

\newcommand{\testLineMark}[1]{%
	\addplot+[%
		forget plot,
% 		opacity=0.5,
		mark=o,
		mark last*,
		mark size=6pt,
	] table[%
		x=TimeInSeconds, 
		col sep=comma,
% 		y=MsUntilRenderComplete,
		y=MsBetweenPresents,
% 		y expr= (1000/(\thisrowno{11})),
	] {#1};
}

%1: input table, 2: input column, 3: output table, 4: output column
\newcommand{\copycolumn}[4]{
\pgfplotstablecreatecol[                                                                                                
         col sep=comma,
         create col/copy column from table={#1}{#2}                                                      
         ]{#4}{#3}   
}

\newcommand*{\ReadOutElement}[4]{%
    \pgfplotstablegetelem{#2}{#3}\of{#1}%
    \let#4\pgfplotsretval
}

\newcommand{\getVal}[2]{%
	\pgfplotstablegetelement{[index]1}{#2}\of{#1}
}

\def\Scene{Arcade}
% \def\Scene{Temple}
% \def\Scene{Bistro}

\begin{tikzpicture}

\begin{axis}[%
	cycle list/Dark2,
% 	cycle multi list={
% 		red,blue,green,orange,black\nextlist
% 		mark=*,mark=x,\nextlist
% 		solid,solid
% 	},
    %only marks,
%     width=\figWidth,
	width=14cm,
	height=10cm,
    %width=20cm,
    % scatter,
    %width=\textwidth,
%     axis y line*=left,
%     y dir=reverse,
    ymin=0,
    ymax=14, %Arcade
%     ymax=55, %Temple
%     ymax=75, %Bistro
    xmin=0,
    xmax=9,
    %>=stealth,
    xlabel={Time},
    x unit=\si{\second},
    ylabel={Frametime (lower is better)},
    y unit=\si{\milli\second},
    legend pos=north east,
    legend style={%
%     	at={(0.5,0.97)},
%     	anchor=north,
    	legend cell align=left, font=\footnotesize},
    legend columns=5,
%     legend entries={$1$, $2$, $3$, $4$, $5$, Baseline, Polarized, Raw Data},
    transpose legend
]
	% Create an intermediate table to read averages from
% 	\pgfplotstableread{avg.dat}\datafile  

	\addlegendimage{empty legend}
	\addlegendentry{}
	\addlegendimage{very thick, dashed, black}
	\addlegendentry{Average polarized}
	\addlegendimage{very thick, black}
	\addlegendentry{Average baseline}
	\addlegendimage{black,opacity=0.3}
	\addlegendentry{Raw data}
	\addlegendimage{empty legend}
	\addlegendentry{}

	\addlegendimage{thick,only marks, Dark2-5-5}
	\addlegendentry{$5$ recursions}
	\addlegendimage{thick,only marks, Dark2-5-4}
	\addlegendentry{$4$ rec.}
	\addlegendimage{thick,only marks, Dark2-5-3}
	\addlegendentry{$3$ rec.}
	\addlegendimage{thick,only marks, Dark2-5-2}
	\addlegendentry{$2$ rec.}
	\addlegendimage{thick,only marks, Dark2-5-1}
	\addlegendentry{$1$ rec.}
	
	
    	\pgfplotsinvokeforeach{5,4,3,2,1}{
    		\foreach\runIt in {1,2,3}{
    			% Plot raw data
    			\testLine{C:/Users/nalo/Desktop/TestsCopy/RenamedFrameTime#1/Baseline\Scene#1_run\runIt_Report.csv}{#1}
			\testLine{C:/Users/nalo/Desktop/TestsCopy/RenamedFrameTime#1/Polarized\Scene#1_run\runIt_Report.csv}{#1}
			
			% Save averages
% 			\copycolumn{C:/Users/nalo/Desktop/TestsCopy/RenamedFrameTime#1/BaselineTemple#1_run\runIt_Log.csv}{RenderAvg}{\datafile}{Baseline#1run\runIt} % make sure column names are unique in output
    		}
    	}
    	\pgfplotsinvokeforeach{5,4,3,2,1}{
		% Average Baseline
		\pgfplotstableread[col sep=comma]{C:/Users/nalo/Desktop/TestsCopy/RenamedFrameTime#1/Baseline\Scene#1_run1_Log.csv}\datafile
    		\pgfplotstablegetelem{0}{RenderAvg}\of\datafile
		\pgfmathsetmacro{\testValA}{1000/\pgfplotsretval}
		\pgfplotstableread[col sep=comma]{C:/Users/nalo/Desktop/TestsCopy/RenamedFrameTime#1/Baseline\Scene#1_run2_Log.csv}\datafile
    		\pgfplotstablegetelem{0}{RenderAvg}\of\datafile
		\pgfmathsetmacro{\testValB}{1000/\pgfplotsretval}
		\pgfplotstableread[col sep=comma]{C:/Users/nalo/Desktop/TestsCopy/RenamedFrameTime#1/Baseline\Scene#1_run3_Log.csv}\datafile
    		\pgfplotstablegetelem{0}{RenderAvg}\of\datafile
		\pgfmathsetmacro{\testValC}{1000/\pgfplotsretval}
		\pgfmathsetmacro{\avgVal}{(\testValA+\testValB+\testValC)/3.0}
		\edef\drawline{\noexpand\draw[very thick,color = Dark2-5-#1] (0, \avgVal) -- (10,\avgVal);}
		\drawline
		
		% Average Polarized
		\pgfplotstableread[col sep=comma]{C:/Users/nalo/Desktop/TestsCopy/RenamedFrameTime#1/Polarized\Scene#1_run1_Log.csv}\datafile
    		\pgfplotstablegetelem{0}{RenderAvg}\of\datafile
		\pgfmathsetmacro{\testValA}{1000/\pgfplotsretval}
		\pgfplotstableread[col sep=comma]{C:/Users/nalo/Desktop/TestsCopy/RenamedFrameTime#1/Polarized\Scene#1_run2_Log.csv}\datafile
    		\pgfplotstablegetelem{0}{RenderAvg}\of\datafile
		\pgfmathsetmacro{\testValB}{1000/\pgfplotsretval}
		\pgfplotstableread[col sep=comma]{C:/Users/nalo/Desktop/TestsCopy/RenamedFrameTime#1/Polarized\Scene#1_run3_Log.csv}\datafile
    		\pgfplotstablegetelem{0}{RenderAvg}\of\datafile
		\pgfmathsetmacro{\testValC}{1000/\pgfplotsretval}
		\pgfmathsetmacro{\avgVal}{(\testValA+\testValB+\testValC)/3.0}
		\edef\drawline{\noexpand\draw[very thick,dashed,color = Dark2-5-#1] (0, \avgVal) -- (10,\avgVal);}
		\drawline
	
    	}

\end{axis}

\end{tikzpicture}

\end{document}